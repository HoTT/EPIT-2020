\documentclass{beamer}
\usetheme{Boadilla}

\usepackage[utf8]{inputenc}
\usepackage{listings}
\usepackage{color}
\usepackage{url}
\usepackage{amssymb}
\usepackage{amsmath}
\usepackage{graphicx}
\usepackage{wrapfig}
\usepackage{mathpartir}
\usepackage{cancel}

\usepackage{wasysym}
\usepackage{tikz}
\usepackage{tikz-cd}
\usetikzlibrary{arrows,shapes,positioning,calc,scopes,
  fit,automata,decorations.pathmorphing,matrix}


\usepackage{microtype}
\DisableLigatures[-]{family=tt*}

\RequirePackage[T1]{fontenc}
\RequirePackage[tt=false,type1=true]{libertine}
\RequirePackage[varqu]{zi4}
\RequirePackage[libertine]{newtxmath}

\setbeamertemplate{navigation symbols}{}

%% Agda stuff
\usepackage{agda}
\newcommand{\anum}[1]{\AgdaNumber{#1}}
\newcommand{\symb}[1]{\AgdaSymbol{#1}}
\newcommand{\data}[1]{{\AgdaDatatype{#1}}}
\newcommand{\record}[1]{{\AgdaRecord{#1}}}
\newcommand{\field}[1]{{\AgdaField{\ensuremath{\mathsf{#1}}}}}
\newcommand{\proj}[1]{\;.\field{#1}}
\newcommand{\func}[1]{{\AgdaFunction{#1}}}
\newcommand{\prim}[1]{{\AgdaPrimitive{#1}}}
\newcommand{\primty}[1]{{\AgdaPrimitiveType{#1}}}
\newcommand{\II}[0]{{\primty{\ensuremath{\mathbb{I}}}}}
\newcommand{\iz}[0]{\con{i0}}
\newcommand{\io}[0]{\con{i1}}
\newcommand{\hcomp}{\prim{hcomp}}
\newcommand{\hcompcon}{\con{hcomp}}
\newcommand{\var}[1]{{\AgdaBound{#1}}}
\newcommand{\keyw}[1]{{\AgdaKeyword{#1}}}
\newcommand{\con}[1]{{\AgdaInductiveConstructor{\ensuremath{\mathsf{#1}}}}}
\usepackage{catchfilebetweentags}
%% \usepackage{MnSymbol}
\usepackage{newunicodechar}
\newunicodechar{≃}{\ensuremath{\simeq}}
\newunicodechar{≅}{\ensuremath{\cong}}
\newunicodechar{∈}{\ensuremath{\in}}
\newunicodechar{⁻}{\ensuremath{^{-}}}
\newunicodechar{ᴹ}{\ensuremath{_{M}}}
\newunicodechar{ᴺ}{\ensuremath{_{N}}}
\newunicodechar{≡}{\ensuremath{\equiv}}
\newunicodechar{λ}{\ensuremath{\lambda}}
\newunicodechar{⊎}{\ensuremath{\uplus}}
\newunicodechar{∷}{\ensuremath{::}}
\newunicodechar{ℓ}{\ensuremath{\ell}}
%% \newunicodechar{₀}{\ensuremath{_0}}
%% \newunicodechar{₁}{\ensuremath{_1}}
%% \newunicodechar{₂}{\ensuremath{_2}}
\newunicodechar{ᵢ}{\ensuremath{_i}}
\newunicodechar{⟨}{\ensuremath{\langle}}
\newunicodechar{⟩}{\ensuremath{\rangle}}
\newunicodechar{α}{\ensuremath{\alpha}}
\newunicodechar{β}{\ensuremath{\beta}}
\newunicodechar{θ}{\ensuremath{\theta}}
\newunicodechar{φ}{\ensuremath{\varphi}}
\newunicodechar{ψ}{\ensuremath{\psi}}
\newunicodechar{η}{\ensuremath{\eta}}
\newunicodechar{ε}{\ensuremath{\varepsilon}}
\newunicodechar{ι}{\ensuremath{\iota}}
\newunicodechar{Σ}{\ensuremath{\Sigma}}
\newunicodechar{∀}{\ensuremath{\forall}}
\newunicodechar{ℕ}{\ensuremath{\mathbb{N}}}
\newunicodechar{→}{\ensuremath{\to}}
\newunicodechar{⊎}{\ensuremath{\uplus}}
\newunicodechar{⋆}{\ensuremath{*}}
\newunicodechar{¬}{\ensuremath{\lnot}}
\newunicodechar{Θ}{\ensuremath{\theta}}

\newcommand{\Agda}{{\tt Agda}}
\newcommand{\CubicalAgda}{{\tt Cubical} {\tt Agda}}
\definecolor{Revolutionary}{RGB}{232,70,68}
\newcommand{\redtt}{\textbf{\texttt{{\color{Revolutionary}red}tt}}}
\newcommand{\RedPRL}{\textbf{\texttt{{\color{Revolutionary}Red}PRL}}}
\newcommand{\cooltt}{\textbf{\texttt{{\color{blue}cool}tt}}}
\newcommand{\coker}{\mathrm{coker}}
\newcommand{\ie}{{\em i.e.}}
\newcommand{\eg}{{\em e.g.}}
\newcommand{\bigO}[1]{\mathcal{O}(#1)}
\let\L=\lstinline

\usepackage{manfnt}
\usepackage{scalerel,stackengine}
\stackMath
\newcommand\reallywidehat[1]{%
\savestack{\tmpbox}{\stretchto{%
  \scaleto{%
    \scalerel*[\widthof{\ensuremath{#1}}]{\kern-.6pt\bigwedge\kern-.6pt}%
    {\rule[-\textheight/2]{1ex}{\textheight}}%WIDTH-LIMITED BIG WEDGE
  }{\textheight}%
}{0.5ex}}%
\stackon[1pt]{#1}{\tmpbox}%
}
\parskip 1ex
\newcommand{\Set}{\mathbf{Set}}
\newcommand{\op}{\mathrm{op}}

% This is a hack to allow matrices in tikz-cd diagrams
\let\amsamp=&

\begingroup
\catcode`\&=13
\gdef\pampmatrix{%
  \begingroup
  \let&=\amsamp
  \begin{pmatrix}%
2}
\gdef\endpampmatrix{\end{pmatrix}\endgroup}
\endgroup


\setbeamertemplate{navigation symbols}{}

\colorlet{beamer@blendedblue}{green!40!black}

\title[\CubicalAgda{}]{Cubical Type Theory and Cubical Agda}

\author[Anders Mörtberg]{{\large Anders Mörtberg}}

\institute[]{\includegraphics[trim=0 2.3cm 0 0,scale=0.2]{su}\\\ \\\ \\}

\date[September 17, 2020]{EPIT -- April 15, \cancel{2020} \textcolor{red}{2021}}

\makeatletter
\newcommand\MyInfo[1]{%
\setbeamertemplate{footline}
{
  \leavevmode%
  \hbox{%
  \begin{beamercolorbox}[wd=.333333\paperwidth,ht=2.25ex,dp=1ex,center]{author in head/foot}%
    \usebeamerfont{author in head/foot}\insertshortauthor
  \end{beamercolorbox}%
  \begin{beamercolorbox}[wd=.333333\paperwidth,ht=2.25ex,dp=1ex,center]{title in head/foot}%
    \usebeamerfont{title in head/foot}#1
      \end{beamercolorbox}%
  \begin{beamercolorbox}[wd=.333333\paperwidth,ht=2.25ex,dp=1ex,right]{date in head/foot}%
    \usebeamerfont{date in head/foot}\insertshortdate{}\hspace*{2em}
    \insertframenumber{} / \inserttotalframenumber\hspace*{2ex}
  \end{beamercolorbox}}%
  \vskip0pt%
}
}
\makeatother

\begin{document}

\begin{frame}[plain]{}
  \titlepage
  \addtocounter{framenumber}{-1}
\end{frame}

\begin{frame}{Identity types}

  The central inductive type in HoTT is the identity type:
  %
  \vspace{-0.4cm}
  \begin{flushleft}
  \ExecuteMetaData[code.tex]{eq}
  \end{flushleft}

  \pause
  This type is crucial to express equations and specifications:
  %
  \begin{mathpar}
    \anum{1}~\func{+}~\anum{1}~\func{≡}~\anum{2}
    \\
    (m~n : \func{ℕ}) \to m~\func{+}~n~\func{≡}~n~\func{+}~m
    \\
    \{A~B : \func{Type}\} → (f : A → B) → \{x~y : A\} → x~\func{≡}~y → f~x~\func{≡}~f~y
    \\
    \hdots
  \end{mathpar}

  Some are provable by \con{refl} because of judgmental/definitional
  equality, some need more elaborate arguments using the
  induction principle \func{J}
\end{frame}

\begin{frame}[fragile]{Identity types}

  \textbf{Pre-HoTT problem:} \func{\_≡\_} is not extensional enough,
  we cannot prove:
  %
  \begin{flushleft}
  \ExecuteMetaData[code.tex]{funExt}
  \end{flushleft}

  It's also impossible to prove propositional extensionality or, more
  generally, univalence. It's also difficult to handle quotient
  types...

  \bigskip

  \only<2->{\textbf{HoTT:} add them as axioms to TT\footnote{Justified by simplicial set model}} \only<3->{\hfill \textbf{breaks canonicity} \frownie{}}

  \bigskip

  \only<4->{\textbf{Cubical:} extend TT and make them provable\footnote{Inspired by \emph{cubical} set model}} \only<5->{\hfill \textbf{preserves canonicity} \smiley{}}
\end{frame}

\begin{frame}{Cubical type theory}

  \textbf{Key idea:} take HoTT very literally and replace inductive
  \func{\_≡\_} with paths

  \pause
  \bigskip

  A path $p : x~\func{≡}~y$ is a function $\var{p} : \func{I} \to A$
  with endpoints $x$ and $y$:
  \begin{mathpar}
    p~\con{i0} = x \and p~\con{i1} = y
  \end{mathpar}

  Get cubes by iteration: $p : \func{I} \to \func{I} \to A$ is a square,
  $q : \func{I} \to \func{I} \to \func{I} \to A$ is a cube, etc...

  \bigskip
  \pause

  This gives us funext, univalence, quotients without sacrificing
  canonicity! %\footnote{But it's harder to justify the induction
  % principle!}

  \bigskip
  \pause

  This simple idea is the basis of \CubicalAgda{}

\end{frame}


\begin{frame}{Cubical proof assistants \hfill~~~~~~~~~~~~~~~~~~~~~~~~~~~~~~~~~~~~~~~~~~~~\manimpossiblecube}

  \CubicalAgda{} was implemented by Andrea Vezzosi, building on a series
  of experimental typecheckers developed at Chalmers: \texttt{cubical},
  \texttt{cubicaltt}...

  \bigskip

  There are also many other cubical and cubically-inspired systems:
  \texttt{Arend}, \RedPRL{}, \redtt{}, \cooltt{}, \texttt{yacctt},
  \texttt{mlang}...

  \bigskip
  \pause

  These build on various cubical models and cubical type theories
  developed by many people over multiple years. The particular flavor
  that \CubicalAgda{} builds on is based on the ``CCHM'' cubical type
  theory of

  \bigskip

  {\small
  \emph{Cubical Type Theory: a constructive interpretation of the
    univalence axiom} (2015) \\
  Cyril Cohen, Thierry Coquand, Simon Huber, Anders Mörtberg \\
  \url{https://arxiv.org/abs/1611.02108}
}

\end{frame}

\begin{frame}{\CubicalAgda{} \hfill~~~~~~~~~~~~~~~~~~~~~~~~~~~~~~~~~~~~~~~~~~~~~~~~~~~~~~~~~~~~~~\manimpossiblecube}

  New features that we will look closer at today:
  \begin{itemize}
  \item Interval (pre-)type \func{I} with endpoints $\con{i0} :
    \func{I}$ and $\con{i1} : \func{I}$
  \item Kan operations (\func{transp} and \func{hcomp})
  \item Computational univalence (via \func{Glue} types)
  \item General schema for higher inductive types
  \end{itemize}

\end{frame}

\begin{frame}{\CubicalAgda{} \hfill~~~~~~~~~~~~~~~~~~~~~~~~~~~~~~~~~~~~~~~~~~~~~~~~~~~~~~~~~~~~\manimpossiblecube}

  The cubical mode has been part of \Agda{} since version 2.6.0 (April 2019)

  \bigskip

  To activate it just open an \texttt{.agda} file and add
  \[
    \texttt{\{-\# OPTIONS --cubical \#-\}}
  \]

  \bigskip

  \pause

  Since October 2018 Andrea Vezzosi and I have been maintaining the
  \texttt{agda/cubical} library:

  \begin{center}
    \url{https://github.com/agda/cubical/}
  \end{center}

  By now $52$ contributors, $56$k LOC, $500$ files

\end{frame}

\begin{frame}{}
  \begin{center}
    {\Huge \CubicalAgda{} time! \\ \ \\}
    % {\footnotesize \url{https://github.com/agda/cubical/blob/master/Cubical/Talks/EPA2020.agda} }
  \end{center}
\end{frame}

\end{document}
